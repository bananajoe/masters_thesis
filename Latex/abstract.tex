\chapter*{Abstract}
\label{cha:abstract}

Consider a sports competition where a number of teams compete with each other within a championship ranking. In each match two teams face each other and receive points depending on the outcome. After a certain amount of matches being played, one can have a look at the current standing and ask questions like.
\begin{itemize}
\item Has team X already won the championship? 
\item Is there a possibility that team Y can overtake its rivals which are ranked higher? 
\item Could it happen that team Z becomes last? 
\end{itemize}

Such questions are referred to as sports elimination problems. It turns out that the hardness of sports elimination problems depends heavily on the rule for receiving points. Soccer, for example, is harder than chess and baseball is probably the easiest case to handle.

We start with providing some background on the origin of sports elimination problems and some basic definitions. In Section 2 the so called \textit{baseball elimination problem} is introduced. For many problems in this set of problems, polynomial time algorithms are known. This thesis treats the \textit{first place elimination} problem providing two different approaches, and the \textit{wildcard elimination problem}. In the last section more general sports competitions are considered. An explanation why some problems are harder than others will be given. The thesis ends with a look on a special case of general sports competition, namely soccer. We treat the \textit{guaranteed point placement} problem where we want to see if one can ensure a fixed team a place among the best $K$ teams for a fixed $K$. 

Subsection 2.1 is based on the paper by Schwartz [1], the paper by Wayne [2] is the basis for Subsection 2.2 and the last subsection for the baseball elimination problem is based on the paper by Adler et. al. [3]. Furthermore Kern, Paulusma are the authors for the paper [4] on which Subsection 3.1 is based on. Last but not least, the final Subsection 3.2 is using the  results of the paper by Christensen, Knudsen and Larsen [5].