\chapter{Introduction}
Different areas of research have to compare trees at some point. Bioinformatics need a measure for comparing the similarities between different RNA-structures and computer scientists have to compare strucured text databases or natural languages. The prelimitariies and circumstances differ resulting in a variety of possibilities of comparing trees. Some techniques can only handle a very specific type of tree, others may work for arbitrary trees, but have disadvantages for special cases. \\
This thesis shall give the reader an inside about multiple tools for comparing trees. We will present a (short) historic background, usabilities, advantages and disadvantages for most comparing techniques. In the last chapter we will compare the two most common comparing techniques with each other. One of them is specialized on a certain type of problem, the other one is an allrounder which can be used on basically any type of tree. 

Comparing trees is a very interesting topic, as it is hard to define general rules of similarity. An observers eye may think that two trees look very similar, but the distance between them may be quite big. Any comparing technique has to put some punishment on dissimilarities. Such dissimilarities might be differing left-to-right ordering of nodes, false labels or the wrong underlying structure among others. 

My motivation for this master thesis was my passion for graph theory and algorithmic thinking as well as my search for a topic that gets applied in the modern day research.  My masters' seminar work on the topic of the tree edit distance~\cite{And} built a solid foundation for this thesis, as it already dealt with the most commonly used and most flexible comparing technique. 